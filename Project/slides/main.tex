\documentclass{beamer}
\usepackage{amsmath, amsfonts, amssymb}
%\usepackage{graphicx}
%\usepackage{pgfplots, pgfplotstable}
\mode<presentation>
{\usetheme{boxes}}
\setbeamertemplate{frametitle}[default][center] 

\usepackage{lmodern, exscale}

%\usepackage{xcolor}
%\usepackage{tikz}
%\usetikzlibrary{shapes, arrows}

%\newcommand{\RR}{\mathbb{R}}
%\newcommand{\eps}{\epsilon}
%\newcommand{\lmat}{\begin{bmatrix}}
%\newcommand{\rmat}{\end{bmatrix}}
%\newcommand{\argmin}{\operatorname{argmin}}
%\newcommand{\fnn}{f_{\mbox{NN}}}
%\newcommand{\tilw}{\tilde{w}}
%\newcommand{\tilW}{\tilde{W}}

\title{Sentiment evolution on important topics from social media}
\author{Devanshu Agrawal}
\date{COSC 690}

\begin{document}

\begin{frame}
\titlepage
\end{frame}

\begin{frame}
\frametitle{Motivation}
\begin{itemize}
\item Understanding attitudes towards important topics of our time is key to communication.
\begin{itemize}
\item Business, politics, art and entertainment, etc.
\end{itemize}
\item Questions:
\begin{itemize}
\item In what topics of conversation are people most engaged?
\item How have sentiments on these topics changed, and where are they headed?
\end{itemize}
\item We assume social media reflects real attitudes.
\begin{itemize}
\item Goal: We address the above questions using a Twitter data set.
\end{itemize}
\end{itemize}
\end{frame}

\begin{frame}
\frametitle{Contributions}
\begin{itemize}
\item We use latent Dirichlet allocation (LDA) to identify the most important topics of conversation on Twitter in the last five years.
\item We record the monthly positive-negative sentiment ratio of tweets in each topic based on emojis.
\item We show that sentiment for each topic exhibits a discernable trend over time and predict future sentiments.
\end{itemize}
\end{frame}

\begin{frame}
\frametitle{Background and related work}
\begin{itemize}
\item Topic modeling.
\item Latent Dirichlet allocation (LDA).
\item Sentiment analysis.
\item Temporal topic-sentiment evolution.
\end{itemize}
\end{frame}

\begin{frame}
\frametitle{Background and related work (cont.)}
\begin{itemize}
\item \textbf{Note to self: } Briefly describe how LDA works.
\end{itemize}
\end{frame}

\begin{frame}
\frametitle{Methodology: Data}
\begin{itemize}
\item We used the Sentiment 140 data set of 1.6 million tweets.
\item Each tweet is given a sentiment score based on its emoji content.
\end{itemize}
\begin{center}
\textbf{Note to self: } Present first few rows of data set here.
\end{center}
\end{frame}

\begin{frame}
\frametitle{Methodology: Preprocessing}
\begin{itemize}
\item For computational efficiency, we selected a subset of $100,000$ tweets based on the distribution of tweets and sentiments over months.
\item We preprocessed each tweet (e.g., removed stop words and punctuation).
\item We trimmed the vocabulary to a manageable size.
\item We formatted all tweets into a TFIDF matrix.
\end{itemize}
\end{frame}

\begin{frame}
\frametitle{Methodology: Topic modeling and sentiment analysis}
\begin{itemize}
\item We applied LDA to the set of tweets.
\item We recorded the positive-negative sentiment ratio of tweets in each topic per month.
\item We performed regression on the sentiments over time and made predictions on the past year.
\item We compared the predictions to the true sentiments from the past year for evaluation.
\end{itemize}
\end{frame}

\begin{frame}
\frametitle{Results: System environment}
\begin{itemize}
\item \textbf{Note to self: } Get Jetstream system information.
\item We implemented our methodology in the Spark framework.
\end{itemize}
\end{frame}

\begin{frame}
\frametitle{Results: Choosing a subset of data}
\begin{itemize}
\item We chose \ldots as our subset of 100,000 tweets.
\end{itemize}
\begin{center}
\textbf{Note to self: } Histograms of tweet count and sentiment.
\end{center}
\end{frame}

\begin{frame}
\frametitle{Results: Key topics}
\begin{itemize}
\item We ran LDA with [list parameter values].
\end{itemize}
\begin{center}
\textbf{Note to self: } Figure of key topics (groups of words) along of percentage of tweets in each topic.
\end{center}
\end{frame}

\begin{frame}
\frametitle{Results: Sentiment evolution}
\begin{itemize}
\item (Some metric of regression for predicted sentiments).
\end{itemize}
\begin{center}
\textbf{Note to self: } Figure of sentiment over time for each topic.
\end{center}
\end{frame}

\begin{frame}
\frametitle{Lessons learned and future work}
\begin{itemize}
\item We wanted to understand how public sentiment changes on key topics of conversation over time.
\begin{itemize}
\item We applied LDA and regression.
\end{itemize}
\item Key topics: [list key topics].
\item Sentiments showed clear trends with [regression fit metric].
\item Next steps:
\begin{itemize}
\item Scale up to all 1.6 million tweets.
\item How do the key topics change over time?
\item Can we extract useful information from hashtags?
\end{itemize}
\end{itemize}
\end{frame}

\end{document}